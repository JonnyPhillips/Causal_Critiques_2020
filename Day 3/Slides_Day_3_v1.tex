% Font options: 10pm, 11pt, 12pt
% Align headings left instead of center: nocenter
\documentclass[xcolor=x11names,compress]{beamer}\usepackage[]{graphicx}\usepackage[]{color}
%% maxwidth is the original width if it is less than linewidth
%% otherwise use linewidth (to make sure the graphics do not exceed the margin)
\makeatletter
\def\maxwidth{ %
  \ifdim\Gin@nat@width>\linewidth
    \linewidth
  \else
    \Gin@nat@width
  \fi
}
\makeatother

\definecolor{fgcolor}{rgb}{0.345, 0.345, 0.345}
\newcommand{\hlnum}[1]{\textcolor[rgb]{0.686,0.059,0.569}{#1}}%
\newcommand{\hlstr}[1]{\textcolor[rgb]{0.192,0.494,0.8}{#1}}%
\newcommand{\hlcom}[1]{\textcolor[rgb]{0.678,0.584,0.686}{\textit{#1}}}%
\newcommand{\hlopt}[1]{\textcolor[rgb]{0,0,0}{#1}}%
\newcommand{\hlstd}[1]{\textcolor[rgb]{0.345,0.345,0.345}{#1}}%
\newcommand{\hlkwa}[1]{\textcolor[rgb]{0.161,0.373,0.58}{\textbf{#1}}}%
\newcommand{\hlkwb}[1]{\textcolor[rgb]{0.69,0.353,0.396}{#1}}%
\newcommand{\hlkwc}[1]{\textcolor[rgb]{0.333,0.667,0.333}{#1}}%
\newcommand{\hlkwd}[1]{\textcolor[rgb]{0.737,0.353,0.396}{\textbf{#1}}}%
\let\hlipl\hlkwb

\usepackage{framed}
\makeatletter
\newenvironment{kframe}{%
 \def\at@end@of@kframe{}%
 \ifinner\ifhmode%
  \def\at@end@of@kframe{\end{minipage}}%
  \begin{minipage}{\columnwidth}%
 \fi\fi%
 \def\FrameCommand##1{\hskip\@totalleftmargin \hskip-\fboxsep
 \colorbox{shadecolor}{##1}\hskip-\fboxsep
     % There is no \\@totalrightmargin, so:
     \hskip-\linewidth \hskip-\@totalleftmargin \hskip\columnwidth}%
 \MakeFramed {\advance\hsize-\width
   \@totalleftmargin\z@ \linewidth\hsize
   \@setminipage}}%
 {\par\unskip\endMakeFramed%
 \at@end@of@kframe}
\makeatother

\definecolor{shadecolor}{rgb}{.97, .97, .97}
\definecolor{messagecolor}{rgb}{0, 0, 0}
\definecolor{warningcolor}{rgb}{1, 0, 1}
\definecolor{errorcolor}{rgb}{1, 0, 0}
\newenvironment{knitrout}{}{} % an empty environment to be redefined in TeX

\usepackage{alltt}
%\documentclass[xcolor=x11names,compress,handout]{beamer}
\usepackage[]{graphicx}
\usepackage[]{color}
\usepackage{booktabs}
\usepackage{hyperref}
\usepackage{tikz}
\usepackage{multirow}
\usepackage{dcolumn}
\usepackage{bigstrut}
\usepackage{amsmath} 
\usepackage{xcolor,colortbl}
\usepackage{amssymb}
%\newcommand{\done}{\cellcolor{teal}#1}

%% Beamer Layout %%%%%%%%%%%%%%%%%%%%%%%%%%%%%%%%%%
\useoutertheme[subsection=false,shadow]{miniframes}
\useinnertheme{default}
\usefonttheme{serif}
\usepackage{Arev}
\usepackage{pdfpages}

\setbeamerfont{title like}{shape=\scshape}
\setbeamerfont{frametitle}{shape=\scshape, size=\normalsize}

\definecolor{dkblue}{RGB}{0,0,102}

\setbeamercolor*{lower separation line head}{bg=dkblue} 
\setbeamercolor*{normal text}{fg=black,bg=white} 
\setbeamercolor*{alerted text}{fg=red} 
\setbeamercolor*{example text}{fg=black} 
\setbeamercolor*{structure}{fg=black} 
 
\setbeamercolor*{palette tertiary}{fg=black,bg=black!10} 
\setbeamercolor*{palette quaternary}{fg=black,bg=black!10} 

\renewcommand{\(}{\begin{columns}}
\renewcommand{\)}{\end{columns}}
\newcommand{\<}[1]{\begin{column}{#1}}
\renewcommand{\>}{\end{column}}

\setbeamertemplate{navigation symbols}{} 
\setbeamertemplate{footline}[frame number]
\setbeamertemplate{caption}{\raggedright\insertcaption\par}

\setbeamersize{text margin left=5pt,text margin right=5pt}

%%%%%%%%%%%%%%%%%%%%%%%%%%%%%%%%%%%%%%%%%%%%%%%%%%




\title{Making Causal Critiques}
\subtitle{Day 3 - Assessing Causal Evidence}
\author{Jonathan Phillips}
\IfFileExists{upquote.sty}{\usepackage{upquote}}{}
\begin{document}

\frame{\titlepage}

\section{Introduction}

\begin{frame}
\frametitle{Solving the Problem of Causal Inference}
\begin{itemize}
\item We cannot!
\item But we can try and minimize the risks
\item Selecting units that provide appropriate counterfactuals, avoiding:
\begin{itemize}
\item Omitted variable bias
\item Selection Bias
\item Reverse Causation
\end{itemize}
\end{itemize}
\end{frame}

\begin{frame} 
\frametitle{Field Experiments}
\begin{itemize}
\item Field experiments provide confidence because treatment assignment is \textbf{controlled by the researcher}
\item But still take place in real-world environments, so they identify (hopefully) meaningful treatment effects
\end{itemize}
\end{frame}

\begin{frame}
\frametitle{Field Experiments}
\begin{itemize}
\item Why does randomization help us achieve causal inference?
\pause
\begin{itemize}
\item A treatment assignment mechanism that balances potential outcomes
\begin{itemize}
\item Every unit has \textbf{exactly the same} probability of treatment
\item No omitted variable bias
\item No self-selection
\item No reverse causation
\end{itemize}
\end{itemize}
\end{itemize}
\end{frame}

\begin{frame}
\frametitle{Field Experiments}
\begin{itemize}
\item Why does randomization help us achieve causal inference?
\begin{itemize}
\item We want to estimate:
\end{itemize}
\begin{eqnarray}
E(Y_1 - Y_0)
\end{eqnarray}
\pause
\begin{itemize}
\item Our data provides:
\end{itemize}
\begin{eqnarray}
E(Y_1|D=1)\text{ ,  }E(Y_0|D=0)
\end{eqnarray}
\pause
\begin{itemize}
\item With randomization, $Y_1, Y_0 \perp D$:
\end{itemize}
\begin{eqnarray}
E(Y_1|D=1) &=& E(Y_1) \\
E(Y_0|D=0) &=& E(Y_0) \\
E(Y_1|D=1) - E(Y_0|D=0) &=& E(Y_1) - E(Y_0) \\
&=& E(Y_1 - Y_0)
\end{eqnarray}
\end{itemize}
\end{frame}

\begin{frame}
\frametitle{Field Experiments}
\begin{itemize}
\item But these are just \textbf{expectations} (averages)
\pause
\begin{itemize}
\item On average, potential outcomes will be balanced
\item More likely in larger samples
\item We cannot verify potential outcomes
\item But we can assess balance in \textit{observable} covariates
\item What if some covariates are imbalanced? %Expected 1/20. Still need to correct as could be real bias.
\end{itemize}
\end{itemize}
\end{frame}

\begin{frame}
\frametitle{Field Experiments}
\begin{itemize}
\item Analysing field experiments
\begin{itemize}
\item Comparison of means: t-test to test significance
\item Regression achieves the same thing
\begin{itemize}
\item $Y_i \sim \alpha + \beta D_i + \epsilon_i$ 
\item $Y_i= Y_{0i} + (Y_{1i} - Y_{0i}) D_i + \epsilon_i$
\item Just the conditional expectation function: $E(Y|D=d)$
\end{itemize}
\item Include covariates if:
\begin{itemize}
\item There is residual imbalance
\item To increase precision of standard errors
\end{itemize}
\end{itemize}
\end{itemize}
\end{frame}

\begin{frame}
\frametitle{Field Experiments}
\begin{itemize}
\item Assumptions
\begin{itemize}
\item \textbf{Compliance with randomization} - Treatment was truly random and accepted
\item \textbf{SUTVA} - Treatment of one unit doesn't affect potential outcomes of other units
\item \textbf{Excludability} - Effects of treatment assignment operate \textbf{only} through treatment
\begin{itemize}
\item Depends if these effects are part of the causal chain
\end{itemize}
\end{itemize}
\end{itemize}
\end{frame}

\begin{frame}
\frametitle{Field Experiments}
\begin{itemize}
\item Limitations of Field Experiments: \textbf{Answerable Questions}
\pause
\begin{itemize}
\item Small sample sizes still prevent inference
\item Ethics
\item Logistics/Finance
\item Some treatments can't be manipulated (history)
\item Lack of control over treatment content and context - is it informative?
\item Long-term effects/adaptation?
\end{itemize}
\end{itemize}
\end{frame}

\begin{frame}
\frametitle{Field Experiments}
\begin{itemize}
\item Limitations of Field Experiments: \textbf{Internal Validity}
\pause
\begin{itemize}
\item No guarantee of actual balance (and Inefficient if we already know confounders)
\item Hawthorne effect: participants adapt behaviour in experiments
\item Biased measurement if not double-blind (non-excludability)
\item Average Treatment Effect can be skewed by Outliers
\item Always complications of non-compliance, SUTVA, attrition
\item Publication/Selection bias
\item Unbiased but imprecise; variation still high if lots of other variables also affect Y
\item Treatment assignment mechanism itself affects outcomes
\end{itemize}
\end{itemize}
\end{frame}

\begin{frame}
\frametitle{Field Experiments}
\begin{itemize}
\item All these complications mean we need lots of assumptions and background knowledge
\item Just as with other methodologies
\end{itemize}
\end{frame}

\begin{frame}
\frametitle{Lab Experiments}
\begin{itemize}
\item Causal Inference
\pause
\item Why lab experiments?
\pause
\begin{itemize}
\item Treatments we cannot administer in reality
\item Outcome measurements that are hard to take in reality
\item Random treatment assignment not permitted in reality
\end{itemize}
\end{itemize}
\end{frame}

\begin{frame}
\frametitle{Lab Experiments}
\begin{itemize}
\item \textbf{Treatment Assignment}: Same as a Field Experiment
\pause
\item \textbf{Treatment}: Not a manipulation of real world political or economic processes, but establishing controlled 'lab' conditions
\pause
\begin{itemize}
\item The advantage: Control over context helps isolate mechanisms
\item The disadvantage: Can we generalize to the real world from this artificial context?
\end{itemize}
\end{itemize}
\end{frame}

\begin{frame}
\frametitle{Natural Experiments}
\begin{itemize}
\item What is a natural experiment?
\pause
\begin{itemize}
\item Treatment assignment is independent of potential outcomes
\begin{itemize}
\item So randomized or 'as-if' random ('exogenous')
\end{itemize}
\pause
\item BUT The researcher doesn't control the treatment assignment process or treatment itself
\begin{itemize}
\item So not a field experiment
\end{itemize}
\item Can make possible analysis of questions that researchers might find unethical or impractical
\end{itemize}
\end{itemize}
\end{frame}

\begin{frame}
\frametitle{Natural Experiments}
\begin{table}[htbp]
  \centering
  \caption{Analysis Types and Assumptions}
    \resizebox*{1.1\textheight}{!}{\begin{tabular}{|r|l|p{2.5cm}|p{2.5cm}|p{2.5cm}|p{6cm}|}
    \hline
    \multicolumn{1}{|r|}{\textbf{Week}} & \multicolumn{1}{l|}{\textbf{Assumption:
}} & \textbf{Researcher Controls Treatment Assignment?} & \textbf{Treatment Assignment Independent of Potential Outcomes} & \textbf{SUTVA} & \multicolumn{1}{p{2cm}|}{\textbf{Additional Assumptions}} \bigstrut\\
    \hline
          & \textbf{Controlled Experiments} &       &       &       &  \bigstrut\\
    \hline
    1     &    Field Experiments & \checkmark & \checkmark & \checkmark &  \bigstrut\\
    \hline
    2     &    Survey and Lab Experiments &  \checkmark & \checkmark & \checkmark & Controlled Environment for treatment exposure \bigstrut\\
    \hline
          & \textbf{Natural Experiments} &       &       &       &  \bigstrut\\
    \hline
    3     &    Randomized Natural Experiments & X     & \checkmark & \checkmark &  \bigstrut\\
    \hline
    4     &    Instrumental Variables & X     & \checkmark & \checkmark & First stage and Exclusion Restriction (Instrument explains treatment but not outcome) \bigstrut\\
    \hline
    5     &    Regression Discontinuity & X     & \checkmark & \checkmark & Continuity of covariates; No manipulation; No compounding discontinuities \bigstrut\\
    \hline
          & \textbf{Observational Studies} &       &       &       &  \bigstrut\\
    \hline
    6     &    Difference-in-Differences & X     & X     & \checkmark & No Time-varying confounders; Parallel Trends \bigstrut\\
    \hline
    7     &    Controlling for Confounding & X     & X     & \checkmark & Blocking all Back-door paths \bigstrut\\
    \hline
    8     &    Matching & X     & X     & \checkmark & Overlap in sample characteristics \bigstrut\\
    \hline
    \end{tabular}}%
\end{table}%
\end{frame}

\begin{frame}
\frametitle{Natural Experiments}
\begin{itemize}
\item Three types of natural experiments
\begin{itemize}
\item 'Pure' natural experiments, where policy is as-if random
\item Instrumental Variables
\item Regression Discontinuities
\end{itemize}
\end{itemize}
\end{frame}

\begin{frame}
\frametitle{Natural Experiments}
\begin{itemize}
\item Because we don't control assignment, we need to verify the assumptions behind natural experiments
\begin{itemize}
\item How do we know assignment was truly random?
\item How was the treatment applied? Consistently?
\end{itemize}
\item We need 'Causal-process observations'
\end{itemize}
\end{frame}

\begin{frame}
\frametitle{Natural Experiments}
\begin{itemize}
\item Challenges due to lack of control over treatment:
\pause
\begin{itemize}
\item We must be lucky to 'find' natural experiments; what if the treatments/experiments that exist don't answer useful political economy questions?
\item The treatment and control groups produced by 'nature' may not produce treatment and control groups which differ in ways that represent a causal effect of interest (Sekhon and Titiunik 2012)
\item We also must be lucky to find a sample that is relevant and interesting - unlike a controlled trial we don't control the recipients either (eg. if we care about states, not municipalities, the audits are no use)
\end{itemize}
\end{itemize}
\end{frame}

\begin{frame}
\frametitle{Natural Experiments}
\begin{itemize}
\item Challenges due to lack of control over treatment: 
\begin{itemize}
\item Spillovers can be an issue - treatment units affect control units' potential outcomes (eg. women's quotas discourage women in non-reserved seats)
\item Generalizability a very open question; what causal process does the experiment really capture?
\item The treatment assignment of a natural experiment might have unique effects (excludability)
\end{itemize}
\end{itemize}
\end{frame}



\begin{frame}
\frametitle{Instrumental Variables}
\begin{itemize}
\item What can we do when the treatment assignment mechanism is not 'as-if' random?
\pause
\item Natural experiments focus on a specific \textbf{part} of treatment assignment that is 'as-if' random
\pause
\item An 'instrument' is a variable which assigns treatment in an 'as-if' random way
\pause
\begin{itemize}
\item Or at least in a way which is 'exogenous' - not related to confounders
\item Even if other confounding variables \textbf{also} affect treatment
\end{itemize}
\end{itemize}
\end{frame}

\begin{frame}
\frametitle{Instrumental Variables}
\begin{itemize}
\item We can use the instrument to isolate 'as-if' random variation in treatment, and use that to estimate the effect of treatment on the outcome
\pause
\item NOT the effect of the instrument on the outcome
\end{itemize}
\end{frame}

\begin{frame}
\frametitle{Instrumental Variables}
\begin{itemize}
\item Example Instruments:
\begin{itemize}
\item Rainfall for conflict 
\item Sex-composition for effect of third child
\item Distance from the coast for exposure to slave trade
\end{itemize}
\end{itemize}
\end{frame}

\begin{frame}
\frametitle{Instrumental Variables}
\begin{itemize}
\item Instrumental Variables Assumptions
\begin{itemize}
\item \textbf{Strong First Stage:} The Instrument must \textbf{affect} the treatment
\pause
\item We can test this with a simple regression: $Treatment \sim Instrument$
\pause
\item The instrument should be a significant predictor of treatment
\item Rule-of-thumb: $F-statistic > 10$
\end{itemize}
\end{itemize}
\end{frame}

\begin{frame}
\frametitle{Instrumental Variables}
\begin{itemize}
\item Instrumental Variables Assumptions:
\begin{itemize}
\item \textbf{Exclusion Restriction:} The Instrument \textbf{ONLY} affects the outcome through its effect on treatment, and not directly
\pause
\item Formally, $cov(Instrument,\text{errors in main regression Y }\sim D)=0$
\pause
\item \textbf{We cannot test or prove this assumption!}
\pause
\item Theory and qualitative evidence needed to argue that the instrument is not correlated with any other factors affecting the outcome
\item Sometimes, the exclusion restriction may be more credible if we include controls
\end{itemize}
\end{itemize}
\end{frame}

\begin{frame}
\frametitle{Instrumental Variables}
\begin{itemize}
\item Instrumental Variables Methodology:
\pause
\begin{enumerate}
\item Use an all-in-one package, eg. \textit{ivreg} in the \textit{AER} package
\begin{itemize}
\item Specify the formula: $Y ~ D | Instrument$
\pause
\end{itemize}
\item Conduct 2-Stage Least Squares: 
\pause
\begin{itemize}
\item Isolate the variation in treatment caused by the instrument: $D \sim Instrument$
\pause
\item Save the predicted values from this regression: $\hat{D} = D \sim Instrument$
\pause
\item Estimate how the predicted values affect the outcome: $Y \sim \hat{D}$
\pause
\item Interpret the coefficient on $\hat{D}$
\end{itemize}
\end{enumerate}
\end{itemize}
\end{frame}

\begin{frame}
\frametitle{Instrumental Variables}
\begin{itemize}
\item IV Interpretation:
\pause
\begin{itemize}
\item Your coefficient is a causal estimate ONLY for units that were actually treated \textbf{because of the instrument}
\pause
\item They don't tell us about the causal effect for other units that never responded to the instrument
\pause
\item We call our causal effect estimate a 'Local Average Treatment Effect' (LATE)
\item 'Local' to the units whose treatment status actually changed
\pause
\end{itemize}
\item Remember, those 'Local' units are not representative so we can't generalize
\end{itemize}
\end{frame}



\end{document}
 
 % effects of causes vs. reverse
